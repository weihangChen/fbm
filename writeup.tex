\documentclass[12pt]{article}
\usepackage{setspace}
\usepackage{fancyhdr}
\usepackage{lastpage}
\usepackage{extramarks}
\usepackage{chngpage}
\usepackage{soul,color}
\usepackage{graphicx,float,wrapfig}
\usepackage{amsmath,amsfonts,amsthm,amssymb,amstext}

\newcommand{\hmwkTitle}{Fractional Brownian Motion Project}
\newcommand{\hmwkDueDate}{}
\newcommand{\hmwkClass}{Capstone}
\newcommand{\hmwkClassTime}{}
\newcommand{\hmwkClassInstructor}{Professor Gatheral}
\newcommand{\hmwkAuthorName}{Yike Lu}

\topmargin=-0.45in      %
\evensidemargin=0in     %
\oddsidemargin=0in      %
\textwidth=6.5in        %
\textheight=9.5in       %
\headsep=0.25in         %

% Setup the header and footer
\pagestyle{fancy}                                                       %
\lhead{\hmwkAuthorName}                                                 %
\chead{\hmwkClass\ \hmwkTitle}  %
\rhead{\firstxmark}                                                     %
\lfoot{\lastxmark}                                                      %
\cfoot{}                                                                %
\rfoot{Page\ \thepage\ of\ \protect\pageref{LastPage}}                          %
\renewcommand\headrulewidth{0.4pt}                                      %
\renewcommand\footrulewidth{0.4pt}                                      %

\newcommand{\ket}[1]{|#1\rangle}
\newcommand{\bra}[1]{\langle #1 |}
\newcommand{\braket}[2]{\langle #1 | #2 \rangle}
\newcommand{\expectation}[1]{\langle #1 \rangle}
\newcommand{\partials}[2]{\frac{\partial #1}{\partial #2}}
\newcommand{\derivs}[2]{\frac{d #1}{d #2}}
\newcommand{\p}[2]{\frac{\partial #1}{\partial #2}}
\newcommand{\curl}[1]{\nabla \times \bf{#1}}
\newcommand{\divc}[1]{\nabla \cdot \bf{#1}}
\newcommand{\ms}[1]{\Delta #1}
\newcommand{\mb}[1]{\mathbf{#1}}
\newcommand{\sgn}[1]{\text{sgn} #1}

\theoremstyle{definition}
\newtheorem{example}{Example}

\numberwithin{equation}{section}
\numberwithin{example}{section}

\begin{document}
\begin{section}{Introduction}
	We seek to backtest simple technical trading rules on Fractional Brownian
	Motions with varying $H$ exponents in order to asses the usefulness of
	$H$ as an input to a trading system.
\end{section}

\begin{section}{Methodology}
	We shall use Monte Carlo. The primary reason for this is that
	we have strict control over $H$, whereas with real market data, we must rely
	on estimators, which introduces the dimension of estimator reliability
	and proper lookback periods for these estimators. Using Monte Carlo
	circumvents this, as we establish $H$ a priori.

	We generate $100000$ Brownian sample paths, each with $252*30=7560$
	data points, representing $30$ trading years. To make things somewhat
	realistic, we use	an annualized volatility $\sigma = 0.3$.

	For a given Brownian sample path ${dB}_i$, we transform the path so as
	to have	various $H$ exponents. The reason we transform the path rather than
	generating new paths is for pathwise comparability, should we choose to
	analyze trading strategies pathwise.

	To that end, we use the following methodology (from Wikipedia)
	\begin{subsection}{Method One}
		One can simulate sample-paths of an fBm as any Gaussian process 
		of known covariance. Say we aim to have simulated values at
		$t_1, \ldots, t_n$.
		\begin{enumerate}
			\item Form the matrix $\Gamma=\bigl(R(t_i,\, t_j), i,j=1,\ldots,\, n\bigr)$
				where $R(t,s)=(s^{2H}+t^{2H}-|t-s|^{2H})/2$.
			\item Compute a square root of $\Gamma$, say $\sigma$. Use for instance
				the Cholesky decomposition method.
			\item Construct a vector of $n$ numbers drawn according a standard
				Gaussian distribution.
			\item Apply $\sigma$ to this vector yields a sample path of an fBm.
		\end{enumerate}

		Running this in python with unoptimized numpy code, generating the matrix
		takes $422$ seconds, which translates to about $7$ minutes. Most of the
		time is taken by the call to $R(t,s)$, however this only takes
		$3.4 \mu s$ per call, so cannot really be sped up significantly.
	\end{subsection}

	Method two is as follows:
	It is also known that: $B_H (t)=\int_0^t K_H(t,s) dB(s)$
	where $B$ is a standard Brownian motion and: 
	\begin{align}
		K_H(t,s)&=\frac{(t-s)^{H-\frac{1}{2}}}{\Gamma(H+\frac{1}{2})} \;_2F_1
		\left (H-\frac{1}{2}; \frac{1}{2}-H; H+\frac{1}{2}; 1-\frac{t}{s}
		\right)
	\end{align}
	Where $ _2F_1$ is the Euler hypergeometric integral.
	Say we want simulate an fbm at points $0=t_0\cdots t_n = T$.
	\begin{enumerate}
		\item Construct a vector of $n$ numbers drawn according a
			standard Gaussian distribution.
		\item Multiply it component-wise by $\sqrt{T/n}$ to obtain the increments
			of a Brownian motion on $[0,T]$. Denote this vector by
			$ (\delta B_1, \ldots, \delta B_n)$.
		\item For each $ t_j$, compute:
			\begin{align}
				B_H (t_j)&=\frac{n}{T}\sum_{i=0}^{j-1}  \int_{t_i}^{t_{i+1}}
				K_H(t_j, s) ds \delta B_i
			\end{align}
	\end{enumerate}
\end{section}
\end{document}
